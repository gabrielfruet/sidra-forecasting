\documentclass[12pt]{article} \usepackage[utf8]{inputenc}
\usepackage[T1]{fontenc} \usepackage[portuguese]{babel} \usepackage{graphicx}
\usepackage{hyperref} \usepackage{booktabs} \usepackage{amsmath}
\usepackage{geometry} \geometry{a4paper, margin=1in}

\title{Relatório de Análise de Série Temporal no Mercado Imobiliário}
\author{Gabriel Fruet} \date{\today}

\begin{document}

\maketitle

\section{Introdução} Este relatório apresenta uma análise da série temporal da
razão de consumidores entre 38 e 58 anos em relação ao número de empresas
ativas por estado, no período de 2007 a 2022. Utilizamos dados da SCOD Brasil e
do IBGE para realizar esta análise.

\section{Metodologia} \subsection{Coleta de Dados} Os dados das empresas foram
obtidos da Tabela 1757 do SIDRA, utilizando requisições HTTP diretamente, sem o
uso de bibliotecas prontas. A população estimada foi obtida da projeção
populacional disponibilizada pelo IBGE.

\subsection{Análise de Dados} Para a análise, utilizamos métodos de
interpolação para ajustar os dados à faixa etária requerida. A série temporal
foi agrupada por estado para identificar padrões de comportamento.

\section{Resultados} \subsection{Análise Temporal} Apresentamos aqui os
gráficos e tabelas que mostram a evolução da razão de consumidores por estado.

\subsection{Estados Saturados e Oportunidades} Identificamos os estados que
apresentam saturação no mercado e aqueles com maiores oportunidades futuras.

\section{Conclusão} Concluímos com as principais descobertas e recomendações
baseadas na análise dos dados.

\section*{Referências} \begin{itemize} \item SCOD Brasil. Tendências do mercado
    imobiliário para 2023. Disponível em:
    \url{https://scod.com.br/blog/post/tendências-do-mercado-imobiliário-para-2023}
    \item IBGE. Tabela 1757 - Dados gerais das empresas de construção.
        Disponível em: \url{https://apisidra.ibge.gov.br/home/ajuda} \item
IBGE. Projeção da população. Disponível em:
\url{https://www.ibge.gov.br/estatisticas/sociais/populacao/9109-projecao-da-populacao.html}
\end{itemize}

\end{document}
